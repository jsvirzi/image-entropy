% !TEX TS-program = pdflatex
% !TEX encoding = UTF-8 Unicode

% This is a simple template for a LaTeX document using the "article" class.
% See "book", "report", "letter" for other types of document.

\documentclass[11pt]{article} % use larger type; default would be 10pt

\usepackage[utf8]{inputenc} % set input encoding (not needed with XeLaTeX)

%%% Examples of Article customizations
% These packages are optional, depending whether you want the features they provide.
% See the LaTeX Companion or other references for full information.

%%% PAGE DIMENSIONS
\usepackage{geometry} % to change the page dimensions
\geometry{a4paper} % or letterpaper (US) or a5paper or....
% \geometry{margin=2in} % for example, change the margins to 2 inches all round
% \geometry{landscape} % set up the page for landscape
%   read geometry.pdf for detailed page layout information

\usepackage{graphicx} % support the \includegraphics command and options

% \usepackage[parfill]{parskip} % Activate to begin paragraphs with an empty line rather than an indent

%%% PACKAGES
\usepackage{booktabs} % for much better looking tables
\usepackage{array} % for better arrays (eg matrices) in maths
\usepackage{paralist} % very flexible & customisable lists (eg. enumerate/itemize, etc.)
\usepackage{verbatim} % adds environment for commenting out blocks of text & for better verbatim
\usepackage{subfig} % make it possible to include more than one captioned figure/table in a single float
% These packages are all incorporated in the memoir class to one degree or another...

%%% HEADERS & FOOTERS
\usepackage{fancyhdr} % This should be set AFTER setting up the page geometry
\pagestyle{fancy} % options: empty , plain , fancy
\renewcommand{\headrulewidth}{0pt} % customise the layout...
\lhead{}\chead{}\rhead{}
\lfoot{}\cfoot{\thepage}\rfoot{}

%%% SECTION TITLE APPEARANCE
\usepackage{sectsty}
\allsectionsfont{\sffamily\mdseries\upshape} % (See the fntguide.pdf for font help)
% (This matches ConTeXt defaults)

%%% ToC (table of contents) APPEARANCE
\usepackage[nottoc,notlof,notlot]{tocbibind} % Put the bibliography in the ToC
\usepackage[titles,subfigure]{tocloft} % Alter the style of the Table of Contents
\renewcommand{\cftsecfont}{\rmfamily\mdseries\upshape}
\renewcommand{\cftsecpagefont}{\rmfamily\mdseries\upshape} % No bold!

%%% END Article customizations

%%% The "real" document content comes below...

\title{Entropy-Based Image Descriminator}
\author{J.S. Virzi}
%\date{} % Activate to display a given date or no date (if empty),
         % otherwise the current date is printed 

\begin{document}
\maketitle

\section{Abstract}

We present a computationally efficient descriminator helpful for finding areas of localized brightness inside a larger image.
Images with clumps of bright pixels on a dim background generally have lower entropy,
as compared to a homogenous image with the same overall brightness.
The lower entropy relates to the higher amount of structure when few pixels contain a disproportionate amount of brightness.
An algorithm of $O \left( N \right)$ is presented where the complexity of intermediate quantities are abstracted away using lookup tables.
Arithmetic complexity is reduced to sums and multiplications.
Iteration of the algorithm at different scales can be employed to find localized structure within a system.
The algorithm is generalized to any system with countable states.

\subsection{Background}

Entropy measures the number of states available to a system. 

For example, consider a gray scale image consisting of ${N = W \times H}$ pixels, where $W$ is the width of the image and $H$ denotes its height. For convenience, assume each pixel is represented by 8 bit integer, encoding 256 different values. 
A completely dark pixel is represented by 0; the brightest value of a pixel is 255.
Let $n_{k}$ represent the brightness of the $k^{th}$ pixel.
The maximum entropy of this image is defined as the logarithm of the number of available states,
assuming each pixel can assume any of 256 values ($0 \le n_{k} \le 255$).

\begin{equation}
S_{max} = \log \left( 256 \right)^{W \times H} = N \times \log \left( 256 \right)
\end{equation}

The total brightness of the image can be considered as the total energy $n \equiv E_{TOT}$ defined as

\begin{equation}
n \equiv \sum_{k=1}^{k=N} n_{k}
\end{equation}

The total number of states available to this image is

\begin{equation}
\sigma = \frac{\left( n_{1} + n_{2} + n_{3} + \ldots + n_{N} \right)!}{n_{1}!n_{2}!n_{3}! \ldots n_{N}!} = \frac{n!}{n_{1}!n_{2}!n_{3}! \ldots n_{N}!}
\end{equation}

The entropy of this image is

\begin{equation}\label{eqn:snom}
% S \equiv \log \left( \sigma \right) = \log \left( \frac{n!}{\Pi_{k=1}^{k=N} \left( n_{k}! \right)} \right)
S \equiv \log \left( \sigma \right) = \log \left( \frac{n!}{n_{1}!n_{2}!n_{3}! \ldots n_{N}!}\right)
\end{equation}

Whereas Eqn.(\ref{eqn1}) assumes the form of the multinomial distribution, the maximum value is assumed when all factors are equal.
From this, we know the maximum entropy of this image is assumed when all pixels have the same value, namely the average energy $\bar{n} \equiv \frac{n}{N}$.
Under these conditions, the entropy is

\begin{equation}\label{eqn:smax}
S_{0} \equiv \log \left( \sigma_{0} \right) = \log \left( \frac{n!}{\left( \bar{n}! \right)^{N}} \right)
\end{equation}

We use Eqn.(\ref{eqn:smax}) to give a reference scale to Eqn.(\ref{eqn:snom})

\begin{equation}
\delta \equiv \frac{\sigma}{\sigma_{0}} =  \frac{\left( \bar{n}! \right)^{N}}{n_{1}!n_{2}!n_{3}! \ldots n_{N}!}
\end{equation}

$\delta$ is a ratio between 0 and 1, comparing the number of available states of the image with energy $n$ to the maximum number of states.
An image with random values of energy (uniform brightness) will have a large entropy, because there are many, many ways to create such an image.
An image with swaths of brightness will have a low entropy, reflecting the much fewer ways to construct such an image.
We investigate the power of this entropy measure to discern if an image might contain otherwise unnaturally clumped swaths of brightness.
It proves convenient to work with the logarithms of these large numbers

\begin{equation}\label{eqn:delta}
\Delta \equiv \log \left( \frac{\sigma}{\sigma_{0}} \right) =  {N * \log \bar{n}!} - {\log n_{1}! - \log n_{2}! - \log n_{3}! - \ldots - \log n_{N}!}
\end{equation}

\section{$\Delta$ as Descriminator}

$\Delta$

\section{Further Development - Clump Detection}

$\Delta$ gives the same value regardless of the distribution of bright pixels.
The pixels could be clumped together, or distributed in any manner across the image, and still give the same value of $\Delta$.
By iterating the algorithm on different scales, effectively grouping pixels together, any clumped structure begins to emerge.
For example, suppose an $100 \times 100$ image contains a $10 \times 10$ clump of brightness in the upper left-hand quadrant.
The upper right-hand quadrant, and both lower quadrants will contain virtually no structure and high entropy.
Only the upper left-hand quadrant will contain structure, and thus low entropy.

\section{Computational Efficiency}

For this example, we have 256 different energies for each pixel.
Computation of $\Delta$ in Eqn. (\ref{eqn:delta}) can be efficiently calculated using a lookup table, containing 256 entries.
The $k^{th}$ entry of the lookup table contains the value $ \lambda_{k} \equiv \log k!$.
Calculation of $\Delta$ involves a sum over pixels.

\section{Generalization of Algorithm}

For the sake of readability, we focused on a particular example.
The methods developed herein are extended, in a straight-forward and self-explanatory manner, to any system with countable states.

\section{Conclusions}

We have developed a descriminator $\Delta$ which can be used as a tool to find localized clumps of structure within a larger system.
Conversely, $\Delta$ can be used to rule out localized structure in an image, or subimage.

\end{document}
